\documentclass[11pt]{article}
\author{Tjaart Bester}
\begin{document}

% some details about the article
\title{COS700 Research Project}
\maketitle


\section{Abstract:}
Cloud computing is a new paradigm that offers a wide variety of  benefits such as high availability, access from anywhere on any device at any time, resources on demand, cost saving, etc. Security concerns have been cited as one of the biggest challenges to this new paradigm. Clarification of these security concerns will help identify real security risks which in turn will help decision maker make more informed decisions about cloud computing. \\

\section{Introduction:}
The increased availability of computing resources in conjunction with the success of the internet has put the cloud computing paradigm at the cutting edge of a digital revolution. Cloud computing offers benefits such as high availability, on demand access from anywhere on any device at any time, cost saving, scalability etc. \cite{Zhang2010} Cloud computing also offers a new business model that outsources computing resources to a shared third party infrastructure. 
\\ \\
Cloud computing resources may be clustered to offer public, private or hybrid networks. Private cloud: also known as internal or corporate cloud. A Private cloud is infrastructure created for a single entity but managed by a third party cloud service provider. Public cloud: is open to the general public over the internet on a pay-per-usage system. Placing systems and critical or sensitive information into a public cloud in the middle of a hostile and open network (the internet) is seen as a risk that most companies are unwilling to take. Hybrid cloud: is a combination of atleast one public and one private cloud.\cite{Onwubiko2010}
\\ \\
Cloud computing services can be classified into infrastructure as a service (IAAS), platform as a service (PAAS) and software as a service (SAAS). Delivering infrastructure, platforms or software as a service requires a high level of virtualisation and implementation of virtual machines. Virtual machines are created, clustered together and operated via a hypervisor. A hypervisor or virtual machine monitor creates a virtual platform for virtual machines and manages the execute of the virtual machines.
\\ \\
Cloud computing offers resources and services to users regardless of physical or geographical boundaries at any time of day or night. For example a pharmaceutical company may move their stock control and manufacturing scheduling systems to a public cloud. This will give their sales representatives across the world access to current stock levels and production schedules in real time. The move to cloud computing will allow greater monitoring and improvement to the company's logistical infrastructure. But the information contained within this system would also be very valuable to competitors trying to secure a competitive edge in the market. Securing such a system with in a public cloud requires risk-based authentication. Risk-based authentication is non-static authentication system that assigns a certain risk profile to each authentication attempt. The risk profile determines the complexity of the challenge. A High risk profile requires a strong challenge and a low risk profile requires a user name and password.

The following section we will look at motivation.
\\ \\

\section{Motivation:}
Authentication on the cloud can be improved by implementing a risk profile for each authentication attempt. For example logging in from your work computer in business hours is a lower risk than logging in from an unknown mobile device in different country in the middle of the night. With a low risk authentication attempt, a normal user name and password will suffice but as the risk increases so does the difficulty of the challenge for example a one time pin, supervisor authorization etc.

The low adoption rate of public clouds indicates that the business world is hesitant to make the move to cloud computing due to perceived security challenges and vulnerabilities. There is a need for cloud computing authentication to be improved and to assure prospective user that cloud computing can give them the competitive edge without unnecessary risk to there business. 

The following section we will look at the problem. 
\\ \\


\section{Problem:}
Taking a look at traditional paradigms in authentication for public or hosted services and resources we see a reliance on physical separation and access only allowed through a fortified proxy. With cloud computing the hosted system or service is no longer within the confines of a company's local network but located in the cloud. Thus we see the need for risk based authentication for resources, information and systems of varying importance within the public cloud is required.  

How to implement risk-based authentication for device independent paradigm such as Cloud Computing? 

The following section we will look at objectives.

\section{Objectives}
Investigate how device identification can be incorporated into user and system authentication within the cloud. \\
Investigate measures to improve user and system authentication on the
cloud.


\bibliography{cos700_2014-05-17}
\bibliographystyle{plain}


\end{document}