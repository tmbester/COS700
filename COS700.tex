\documentclass[11pt]{article}
\author{Tjaart Bester}
\usepackage{harvard}
\begin{document}

% some details about the article
\title{COS700 Research Project}
\maketitle


\section{Abstract:}
The increased availability of computing resources in conjunction with the success of the Internet has put the cloud computing
paradigm at the cutting edge of a digital revolution. Cloud computing offers benefits such as high availability, on demand access from
anywhere on any device at any time, cost saving, scalability and etc. Cloud computing also offers a new business model that outsources
computing resources to a shared third party infrastructure. Cloud computing services can be classified into infrastructure as a service
(IAAS), platform as a service (PAAS) and software as a service (SAAS). Delivering infrastructure, platforms or software as a service
requires a high level of virtualization and implementation of virtual machines. Virtual machines are created, clustered together and
operated via a hyper-visor. A hyper-visor or virtual machine monitor creates a virtual platform for virtual machines and manages the
execution thereof. Cloud computing offers resources and services to users regardless of physical or geographical boundaries at any
time of day or night. However, the cloud as one of the most promising technology developments of this century is hampered by a string
of security challenges which has led to its low adoption rates, more especially for public cloud infrastructure. The low adoption rate
of public clouds indicates that the business world is hesitant to make the necessary move to cloud computing due to perceived security
challenges and vulnerabilities.
Hence, this research proposes a dynamic risk-based authentication mechanism. Securing IT-resources that are hosted on a public cloud
requires risk-based authentication. The proposed solution assigns a certain risk profile to each authentication attempt. The risk
profile determines the complexity of the challenge. A high risk profile requires a strong challenge and a low risk profile requires a
user name and password. This paper argues that authentication on the cloud can be improved by implementing a risk profile for each
authentication attempt, as stipulated above. For example logging in from your work computer in business hours is a lower risk than
logging in from an unknown mobile device in different country in the middle of the night. With a low risk authentication attempt, the
proposed system would require a normal user to authenticate with name and password only and this would suffice for just that particular
scenario. However, as the risk increases so does the difficulty of the authentication challenge; for example, a one-time pin (OTP),
supervisor authorization etc would be required on top of the usual username and password. There is a need to improve authentication
systems in order to assure prospective users that public cloud computing can give them competitive edge without exposing their confidential
data to unnecessary risk of falling into the wrong hands of unauthorized third parties and threatening their business bottom line.\\

\section{Introduction:}
The increased availability of computing resources in conjunction with the success of the internet has put the cloud computing paradigm at the cutting edge of a digital revolution. Cloud computing offers benefits such as high availability, on demand access from anywhere on any device at any time, cost saving, scalability etc. \cite{Zhang2010} Cloud computing also offers a new business model that outsources computing resources to a shared third party infrastructure. 
\\ \\
Cloud computing resources may be clustered to offer public, private or hybrid networks. Private cloud: also known as internal or corporate cloud. A Private cloud is infrastructure created for a single entity but managed by a third party cloud service provider. Public cloud: is open to the general public over the internet on a pay-per-usage system. Placing systems and sensitive information into a public cloud in the middle of a hostile and open network (the internet) is seen as a risk that most companies are unwilling to take. Hybrid cloud: is a combination of atleast one public and one private cloud.\cite{Onwubiko2010}
\\ \\
Cloud computing services can be classified into infrastructure as a service (IAAS), platform as a service (PAAS) and software as a service (SAAS). Delivering infrastructure, platforms or software as a service requires a high level of virtualisation and implementation of virtual machines. Virtual machines are created, clustered together and operated via a hypervisor. A hypervisor or virtual machine monitor creates a virtual platform for virtual machines and manages their execute.
\\ \\
Cloud computing offers resources and services to users regardless of physical or geographical boundaries at any time of day or night. For example a pharmaceutical company may move their stock control and manufacturing scheduling systems to a public cloud. This will give their sales representatives across the world access to current stock levels and production schedules in real time. The move to cloud computing will allow greater monitoring and improvement to the company's logistical infrastructure. But the information contained within this system would also be very valuable to competitors trying to secure a competitive edge in the market. Securing such a system with in a public cloud requires more than single factor authentication. Risk-based authentication is non-static authentication system that assigns a certain risk profile to each authentication attempt. The risk profile determines the complexity of the challenge. A High risk profile requires a strong challenge and a low risk profile can require as little as user name and password.

The following section we will look at motivation.
\\ \\

\section{Motivation:}
Authentication on the cloud can be improved by implementing a risk profile for each authentication attempt. For example logging in from your work computer in business hours is a lower risk than logging in from an unknown mobile device in different country in the middle of the night. With a low risk authentication attempt, a normal user name and password will suffice but as the risk increases so does the difficulty of the challenge for example a one time pin, supervisor authorization etc.

The low adoption rate of public clouds indicates that the business world is hesitant to make the move to cloud computing due to perceived security challenges and vulnerabilities. There is a need for cloud computing authentication to be improved and to assure prospective user that cloud computing can give them the competitive edge without unnecessary risk to there business. \\
The following section we will look at the problem. 
\\ \\


\section{Problem:}
The usage of mobile devices within the cloud computing is set to rise, even within corporate environments with the adoption of BYOD (bring you own device). Combined with the increase of different cloud services supplied by a myriad of service providers, has highlighted the need for a federated identity management with single sign-on service.  \\ \\ 
How to construct a federated identity management service to be used with cloud computing risk-based authentication? \\ \\
The following section we will look at objectives.

\section{Objectives:}
Investigate how device identification can be incorporated into user and system authentication within the cloud. \\
Investigate measures to improve user and system authentication on the
cloud.

\section{Related works:}
This section discusses existing authentication mechanisms for public cloud solutions. This is to demonstrate that this work is grounded on existing work and does not exist in isolation. A number of authors have already made some attempts to address the issue of weak authentication on public cloud infrastructures. For instance, \cite{Archer2009} presents some credentials (for example username and password) which is checked against registered information for said user. Should the credentials match the systems values the user is then deemed authenticated. How ever, should the store containing usernames and passwords be modified or leaked the entire system is endanger of being compromised.   

A. Cecil Donald et al. \cite{A2014} proposes a novel authentication  mechanism to enhance security in the cloud environment. The use of a trusted authority that creates a digital signature to compare with the users created digital signature when services are requested from a cloud service provider. This approach uses the digital signature  contained in the Portuguese identity smart card. However, the challenge with this approach is that a breach or loss of confidentiality (if pin becomes public knowledge) would require the re-issuing of an ID card. This has the potential to render IT resources unavailable or inaccessible at least for the time it takes to issue a new ID card. If not done timeously could result in lost productivity.

Richard Chow et al. \cite{Chow2010} extends the knowledge base and proposes a novel authentication framework for mobile devices to cloud services. Where the recent history and activity is used to determine the appropriate level of authentication required. A Major issue with the high emphasis on the user's mobile device is the risk of make the device an even bigger target for theft and spoofing (as seen with internet banking theft/fraud). The use of SMS history and other information form mobile devices for authentication could be seen as an invasion of the users privacy, possible exposing other unforeseen mobile security issues.

L. B. Jivanadham et al. cite{} proposes an integrated authentication mechanism called cloud cognitive authenticator (CCA). CCA is an application program interface (API) integrating bio-signals, one round Zero Knowledge Protocol (ZKP) for authentication and Rijndael algorithm in Advance Encryption Standard (AES). The novelty of this approaches is in the two levels of authentication. The interoperability of the CCA and AES algorithm compatibility was questioned by A. Cecil Donald cite{A2014}.   

L.F.B Soares et al. \cite{Soares2013} suggests a model where a proxy  VM is placed between inbound connections and cloud management interfaces. This novel approach of compartmentalizing VM's to limit the spread of security breaches. This proposed model strengthens security against inter VM attacks, the model can be extended to include multifactor authentication to improve VM access control.

\section{Prototype}
The prototype uses Xenserver for a hypervisor to supply the platform resource of a public PAAS cloud solution. Access to the cloud is controlled by a vnc-gateway server that send SAML2 authentication assertions to a IDP (WSO2 Identity server). The authentication assertion contains information gather from the connection devise (gps co-ordinates, trace route, etc.) to be compared with a risk profile for the user account.    

\bibliographystyle{plain}
\bibliography{cos700_2014-10-07}



\end{document}